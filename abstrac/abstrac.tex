%\centering

\section*{Abstract} 

The team-building process is complex in multiple areas, from a baseball team to the formation of the faculty who teach a subject. This is due to the large number of combinations of possible assignments, among all the factors to take into account. In the literature there are various investigations about these issues. In particular, there is a work where a model is defined in which the necessary information to manage for the problem of conformation of software teams is reflected. This model takes into account individual and collective factors that contribute to the formation of the team as a whole. In addition, a tool is proposed that provides support to the proposed model. \\

The present work aims to evaluate the relevance of applying the model and the tool to form software teams, in the problems of formation of baseball and teaching teams. In addition, the tool incorporates the functionality of importing real people's data and transforming it into data that can be managed by the tool.

\begin{description}
	\item[Key words:]{team formation, teaching teams, baseball team, optimization.}
\end{description}