%\centering

\section*{Abstract} 

The team-building process is complex in multiple areas, from a baseball team to the formation of the faculty who teach a subject. This is due to the large number of combinations of possible assignments between the available teams, roles and people. In the literature there are various investigations about these issues. In particular, there is a work where a model is defined in which the necessary information to manage for the problem of software team formation is reflected. This model takes into account individual and collective factors that use the formation of the team as a whole. In addition, a tool that supports the proposed model is proposed, which is called TEAMSOFT$^+$. Although the model was conceived before forming software teams, it is configurable, since it can define the competencies and roles to be played, so it is presumed that it can be used in other contexts.\\

The objective of this work is to evaluate the relevance of applying the model and the tool to form software teams, in the formation of baseball teams and teachers to teach, and identify and implement the changes necessary to achieve it. As a result of the work, the function of importing the real data of the people and transforming it into manageable data is incorporated into the tool.
\begin{description}
	\item[Key words:]{team building, baseball teams, teacher teams to teach, optimization.}
\end{description}