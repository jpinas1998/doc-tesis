%\centering

\section*{Abstract} 

The team-building process is complex in multiple areas, from a baseball team to the formation of the faculty who teach a subject. This is due to the large number of combinations of possible assignments between the available teams, roles and people. In the literature there are various investigations about these issues. In particular, there is a work where a model is defined in which the necessary information to manage for the problem of conformation of software teams is reflected. This model takes into account individual and collective factors that contribute to the formation of the team as a whole. In addition, a tool that supports the proposed model is proposed, which is called TEAMSOFT$^ +$. Although the model was initially conceived to form software teams, it is configurable, since the competencies and roles to be played can be defined, so it is presumed that it can be used in other contexts.\\

The present work aims to evaluate the relevance of applying the model and the tool to form software teams, in the problems of baseball team formation and teaching. In addition, the tool incorporates the functionality of importing real people's data and transforming it into data that can be managed by the tool.

\begin{description}
	\item[Key words:]{team formation, teaching teams, baseball team, optimization.}
\end{description}