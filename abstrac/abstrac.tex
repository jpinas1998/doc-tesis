%\centering

\section*{Abstract} 

The team-building process is complex in multiple areas, from a baseball team to the formation of the faculty who teach a subject. This is due to the large number of combinations of possible assignments, among all the factors to take into account. In the literature there are various investigations about these issues, although there are few studies that deal with the formation of the team as a whole. In particular, there is a work where a model is defined in which the necessary information to manage for the problem of conformation of software teams is reflected. This model takes into account individual and collective factors that contribute to the formation of the team as a whole. In addition, a tool is proposed that provides support to the proposed model. The objective of this work is to evaluate the relevance of applying the model and the tool to form software teams in other types of problems. In addition, a tool is developed to import the information stored on people in known databases \footnote {term used to refer to databases in the public domain}, and transform them into data that can be managed by the software equipment conformation system .

\begin{description}
	\item[Key words:]{team Formation, optimization.}
\end{description}