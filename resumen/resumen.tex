\section*{Resumen}
El proceso de formación de equipos resulta complejo en múltiples ámbitos, desde un equipo de béisbol, hasta la formación del claustro de profesores que imparten una asignatura. Esto se debe al gran número de combinaciones de posibles asignaciones, entre todos los factores a tener en cuenta. En la literatura existen diversas investigaciones acerca de estos temas, aunque resultan escasos los trabajos que tratan la formación del equipo como un todo. En particular, existe un trabajo donde se define un modelo en el cual queda plasmada la información necesaria a gestionar para el problema de conformación de equipos de software. Este modelo toma en cuenta factores individuales y colectivos que contribuyen a la formación del equipo como un todo. Además, se propone una herramienta que brinda soporte al modelo propuesto. El presente trabajo tiene como objetivo evaluar la pertinencia de aplicar el modelo y la herramienta para formar equipos de software, en otros tipos de problemas. Además, se desarrolla una herramienta para importar la información almacenada de las personas en bases de datos conocidas, y transformarlas en datos gestionables por el sistema de conformación de equipos de software.

\begin{description}
	\item[Palabras claves:]{conformación de equipos, optimización.}
\end{description}
%\end{abstract}


