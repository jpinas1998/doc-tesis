\section*{Resumen}
El proceso de formación de equipos resulta complejo en múltiples ámbitos, desde un equipo de béisbol, hasta la formación del claustro de profesores que imparten una asignatura. Esto se debe al gran número de combinaciones de posibles asignaciones, entre todos los factores a tener en cuenta. En la literatura existen diversas investigaciones acerca de estos temas. En particular, existe un trabajo donde se define un modelo en el cual queda plasmada la información necesaria a gestionar para el problema de conformación de equipos de software. Este modelo toma en cuenta factores individuales y colectivos que contribuyen a la formación del equipo como un todo. Además, se propone una herramienta que brinda soporte al modelo propuesto. \\

El presente trabajo tiene como objetivo evaluar la pertinencia de aplicar el modelo y la herramienta para formar equipos de software, en los problemas de conformación de equipos de béisbol y docencia. Además, se incorpora a la herramienta la funcionalidad de importar los datos reales de las personas y, transformarlas en datos gestionables por la herramienta.

\begin{description}
	\item[Palabras claves:]{conformación de equipos, equipos de béisbol, equipos de docencia, optimización.}
\end{description}
%\end{abstract}


