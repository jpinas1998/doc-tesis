\chapter*{Introducción}

El proceso de formación de equipos resulta complejo en medianas y grandes organizaciones, debido a la gran cantidad de combinaciones de asignaciones posibles entre la cantidad de equipos, roles y personas disponibles \cite{Mayi09}. Esto hace que sea necesario el uso de herramientas o sistemas informatizados que apoyen la toma de decisiones. Estas herramientas se basan en el uso de modelos matemáticos que representen el problema a resolver lo más objetivamente posible.\\ 

El reto de formar equipos capaces de desarrollar proyectos de software exitosos según \cite{ana14} constituye un problema en el que interviene múltiples factores. Para resolver este problema es necesario tener en cuenta varios aspectos \cite{ana15}:
\begin{itemize}
	
\item Proyectos: conjunto de objetivos relacionados a cumplir por un grupo de personas en un período de tiempo definido.

\item Roles: funciones a cumplir por las personas en un proyecto.

\item Personas: responsables de llevar a cabo las tareas correspondientes a los roles vinculados a un proyecto.

\item Competencias genéricas: características asociadas al comportamiento general de una persona.

\item Competencias técnicas: características asociadas a los conocimientos o habilidades técnicas específicos a un proyecto. 

\item Tipos psicológicos: clasificaciones de las personas según su perfil psicológico.

\item Roles de Belbin: conjunto de roles mentales, sociales y de acción, definidas por Belbin necesarias en un equipo.

\end{itemize}	

En \cite{Mayi09}, se define un modelo donde queda plasmada la información necesaria a gestionar para el problema de formación de equipos de software. Este modelo toma en cuenta factores que contribuyan a la asignación individual a los roles del proyecto y a la formación del equipo como un todo. Además se propone una herramienta denominada: TEAMSOFT$^+$, que brinda soporte a este modelo.\\

Sin embargo, tanto el modelo presentado en \cite{Mayi09} como la herramienta que le brinda soporte, fueron diseñados para formar un solo equipo; esto limita su uso cuando se desean formar múltiples equipos. Utilizar la herramienta desarrollada para dar solución a las situaciones anteriores implicaría formar los equipos de uno en uno (de forma secuencial). Obteniendo como resultado un desbalance entre los primeros y los últimos equipos, dado que en cada iteración se seleccionan los mejores candidatos disponibles. En el año 2018, tomando en cuenta los trabajos desarrollados sobre el tema, se definió un nuevo modelo que permite la formación de múltiples equipos de proyecto \cite{Duran2019}. El modelo toma en cuenta las cuatro funciones objetivos consideradas en la propuesta en \cite{Mayi09} (maximizar competencias, minimizar incompatibilidades, balancear la carga de trabajo y minimizar el costo de desarrollo a distancia) e incorpora funciones como maximizar el interés por desempeñar el rol, maximizar la presencia de roles de Belbin, maximizar la diversidad de tipos psicológicos según el test de Myers-Briggs y maximizar el interés por trabajar en el equipo.\\

Un ejemplo clásico en el deporte es el béisbol, que al ser un deporte de equipo, manifiesta también esta problemática. Es considerado como uno de los deportes más populares en América y Asia, específicamente en países como: Cuba, Japón, Estados Unidos, entre otros. Resulta entonces de gran interés la selección de los jugadores para formar un equipo que cumpla con las expectativas de la dirección del equipo y sus fanáticos. Para esto, el director del equipo tiene en cuenta las habilidades y características propias de cada jugador. No son pocos los casos en los que los directores realizan este proceso de forma manual e intuitiva. En \cite{Smith1995} se realiza un estudio sobre las competencias necesarias a tener en cuenta para la formación del equipo. Sin embargo, en la actualidad, las soluciones existentes para este problema \cite{Polyashuk2015, Sugrue2007} no las tienen en cuenta. En estos trabajos, los autores se basan en estadísticas almacenadas de los jugadores a lo largo de los años para construir un indicador y, en base a este, realizar la asignación. \\

Otra situación en la que está presente la formación de equipos, es a la hora de asignar profesores a los tipos de clases \footnote{Por ejemplo: conferencia, clase práctica, seminario, etc, según \cite{res2018}} que le corresponden a las asignaturas. Muchas universidades del mundo tienen que enfrentar este proceso al menos una vez al año. Se han realizado múltiples investigaciones en la literatura enfocándose en la asignación de los profesores a las asignaturas. Por ejemplo, en \cite{Bosquez2020} se propone un modelo para la asignación de asignaturas a profesores, basándose en la preferencia de los mismos hacia las asignaturas (si le interesaba darla o no). En \cite{Domenech2014} se presenta otro modelo, similar al anterior, pero esta vez los autores deciden realizar este proceso asignando los profesores a las asignaturas. Este último tiene en cuenta el balance de la carga de los profesores y sus preferencias hacia las asignaturas. En ninguno de los trabajos revisados los autores tienen en cuenta para realizar la asignación las competencias de los profesores, ni las competencias necesarias para cada cumplir cada rol.\\

A partir de lo anterior, se puede identificar como \textbf{problema de investigación:} ¿Cómo adaptar los problemas de formación de equipos de de béisbol y docencia, al modelo que le da soporte TEAMSOFT$^+$? Para responder al problema de investigación, se define el siguiente \textbf{objetivo general:} evaluar la pertinencia de aplicar el modelo y la herramienta TEAMSOFT$^+$ que le da soporte, para formar equipos de béisbol y docentes.\\\\
A partir del análisis del objetivo general se derivaron los siguientes \textbf{objetivos específicos y tareas:}
\begin{itemize}		
	\item Identificar los factores a tomar en cuenta en la formación de equipos docentes y de béisbol.
		\begin{itemize}
			\item Analizar investigaciones relacionadas con el tema de la formación de equipos docentes y de béisbol.
			\item Evaluar los modelos propuestos en estos trabajos.
			\item Comparar los modelos correspondientes a los trabajos identificados.
		\end{itemize}
	
	\item Evaluar la pertinencia de aplicar el modelo soportado por TEAMSOFT$^+$ en los problemas de béisbol y docencia.
		\begin{itemize}
			\item Estudio del modelo soportado por TEAMSOFT$^+$.
			\item Representar mediante ejemplos estos problemas utilizando el modelo soportado por TEAMSOFT$^+$.
		\end{itemize}
		
	\item Incorporar una funcionalidad a la herramienta TEAMSOFT$^+$ que permita la obtención de características de las personas a partir de ficheros.
		\begin{itemize}
			\item Caracterizar las fuentes o herramientas existentes que gestionan datos necesarios para la formación de equipos docentes y de béisbol.
			\item Desarrollar una funcionalidad configurable para los problemas de formación de equipos de béisbol y docencia.
			\item Validar de la funcionalidad.
		\end{itemize}
\end{itemize}


El aporte práctico de este trabajo consiste en la incorporación a TEAMSOFT$^+$ de la funcionalidad de importar datos reales de las personas y formar equipos en los contextos de la docencia y el béisbol. \\
%Estos datos se obtienen de sitios públicos. \\

El documento está estructurado en tres capítulos. En el Capítulo \ref{chap:1} se describe la información general que se gestiona en el modelo que le da soporte TEAMSOFT$^+$, así como la modelación de los problemas de béisbol y docencia. En el Capítulo \ref{chap:2} se menciona qué aspectos del modelo soportado por TEAMSOFT$^+$ se utilizan y cuáles no para la modelación de estos problemas. Se explica además, cómo se pueden transformar los datos que se disponen en bases de datos conocidas \citep{DISERTIC2020, INDER2020} en datos entendibles por la herramienta. Para terminar el capítulo, se explica la funcionalidad a incorporar. Por último, en el Capítulo \ref{chap:3},  se realiza la validación de la funcionalidad a incorporar.

