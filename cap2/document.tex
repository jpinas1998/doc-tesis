\documentclass[10pt,a4paper]{article}
\usepackage[latin1]{inputenc}
\usepackage{amsmath}
\usepackage{amsfonts}
\usepackage{amssymb}
\usepackage{graphicx}
\author{test}
\begin{document}
	\begin{eqnarray}
	x=3+4, 56
	\end{eqnarray}
	$\displaystyle{ t_i = \sum_{j=1}^{n_i} c_j*\dfrac{I_j}{R_j}} \; \; \; i \neq 4, 5$
\end{document}